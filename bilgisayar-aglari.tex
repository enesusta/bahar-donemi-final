\documentclass[a4paper,10pt]{article}
\usepackage{hyperref}
\usepackage{amsmath}
\usepackage{graphicx}
\usepackage{mathtools}
\usepackage{enumitem}
\usepackage{multicol}
\usepackage{listings}
\usepackage{booktabs}
\usepackage[turkish]{babel}
\usepackage[utf8]{inputenc}
%\usepackage[T1]{fontenc}
%\usepackage{color}
\usepackage[a4paper,bindingoffset=0.2in,%
            left=0.2in,right=0.2in,top=0.4in,bottom=0.4in,%
            footskip=.25in]{geometry}
%\usepackage{blindtext}
%\usepackage{breqn}
%\usepackage{tabto}
%\usepackage{picture}

\lstset{
   language=Java,
   captionpos=t,
   tabsize=3,
   %frame=single,
   frameround=tttt
   backgroundcolor=\color{highlight},
   basicstyle=\footnotesize\ttfamily,
   keywordstyle=\color{purple}\bfseries,
   commentstyle=\color{gray},
   stringstyle=\color{red},
   morecomment=[s][\color{blue}]{/**}{*/},
   %numbers=left,
   %numberstyle=\tiny,
   %numbersep=5pt,
   breaklines=true,
   showstringspaces=false,
   emph={label},
   inputencoding=utf8,
   extendedchars=true,
   % German umlauts
   literate=%
   {Ö}{{\"O}}1
   {Ä}{{\"A}}1
   {Ü}{{\"U}}1
   {ß}{{\ss}}1
   {ü}{{\"u}}1
   {ä}{{\"a}}1
   {ö}{{\"o}}1
   %Türkçe karakterler
   {ı}{{\i}}1
   {İ}{{\.{I}}}1    % This is the problem character.
   {ğ}{{\u{g}}}1
   {Ğ}{{\u{G}}}1
   {ş}{{\c{s}}}1
   {Ş}{{\c{S}}}1
   {ç}{{\c{c}}}1
   {Ç}{{\c{C}}}1
}
%\definecolor{mauve}{rgb}{0.58,0,0.82}

\begin{document}


  \begin{tabular}{ l | c }
    \hline
    Numara & 1171602069 \\ \hline
    Ad Soyad & Enes Usta \\
    \hline
  \end{tabular}

\begin{center}
\textbf{TRAKYA ÜNİVERSİTESİ} \\
\textbf{BİLGİSAYAR MÜHENDİSLİĞİ BÖLÜMÜ} \\
\textbf{BİLGİSAYAR AĞLARI} \\
\textbf{FİNAL ÖDEVİ}
\end{center}

\begin{center}
\textbf{Not:} Bu döküman \textbf{LaTeX} ile Enes Usta tarafından oluşturulmuştur.
Kaynak kodu için: \href{https://github.com/enesusta/bahar-donemi-final/blob/master/bilgisayar-aglari.tex}{Tıklayınız}.
\end{center}

\textbf{1.} SQL SERVER, TCP/IP protokolünü kullanarak iletişim kuran bağlantı noktası 1433 olan bir
sunucu uygulamasıdır. Aşağıdaki ağ kurulumunda; uzak istemci, geniş alan ağına doğrudan
bağlanan bir bilgisayar veya mobil cihazdır. Uzak istemcinin, LAN’da bulunan SQL SERVER’a
erişebilmesi için soru işareti ile gösterilen cihaz ne olmalıdır? Cihazda hangi servisler
koşmalıdır, hangi ayarlar yapılmalıdır. SQL SERVER cihazında yapılması gereken TCP/IP
ayarları nelerdir? Tüm detaylarıyla yazınız. Örnek konfigürasyonu gerçek adresler/rakamlar
ile anlatınız.


\begin{figure}[htb]
\begin{center}
\includegraphics[]{a1.png}
\end{center}
\caption{Sistem}
\label{fig:Sistem}
\end{figure}

\vspace{0.5cm}

Soru işareti ile gösterilen cihaz \textbf{router} olmalıdır. Uzak bir istemciden erişim sağlamak için SQL Server
cihazında aşağıdaki ayarlamalar yapılmalıdır.

\vspace{1cm}

\textbf{SQL Server Ayarları}

\begin{enumerate}
  \item SQL Server Configuration Manager açılır.
  \item SQL Server Network Configuration altında bulunan
\item Protocols for MSSQLSERVER(sunucu servis adı) üzerine tıklanır.
 \item Ardından TCP/IP protokolüne sağ tıklayıp
özellikler seçilir.
 \item IP Addresses sekmesi altında bulunan IP’lerin Enabled özelliği de Yes olarak ayarlanıp
\item TCP Port’larına 1433 yazılır.
 \item Gerekli ayarlamalar tamamlandığında SQL server servisi yeniden başlatılır.
\end{enumerate}

\pagebreak

\textbf{Güvenlik Duvarı Ayarları}

\vspace{0.5cm}

Windows Güvenlik Duvarı açılıp Advanced Settings seçeneğine tıklanır. Gelen pencerede sol
üstte bulunan Inbound Rules seçeneğine tıklanıp sağ taraftaki New Rule seçilir. Rule Type sekmesinde
Port kısmı seçilir. Protocol and Ports sekmesinde TCP protokolü seçilir ve Specific local ports kısmına 
1433 yazılır. Action sekmesinde Allow the connection denir. Profile sekmesinde uygun bir profil seçilir ve
son olarak da Name sekmesinde bu kurala bir isim verilerek Finish butonuna basılır.

\vspace{0.5cm}

\textbf{SQL Server Management Studio ile Kullanıcı ve Şifre Ayarları}

\vspace{0.5cm}

\textbf{Microsoft SQL Server Management Studio} açılır ve sunucuya bağlanılır. Security bölümü
genişletilerek Logins seçeneğine sağ tıklanır ve \textbf{New Login} seçilir. Böylece SQL Server’ımıza yeni kullanıcı
ekleme ekranına ulaşılır. Login name kısmında yeni kullanıcı ismi oluşturulur. \textbf{Windows Authentication}
seçeneği yerine \textbf{SQL Server Authentication} seçeneği seçilir ve parola oluşturulur. 

\vspace{0.2cm}
\textbf{Enforce Password Policies} seçimi kaldırılır. Pencerenin sol üst kısmında bulunan \textbf{User Mapping}
bölümüne gidilir. Açılan sayfada kullanıcının erişmesi için gerekli veri tabanları işaretlenir ve işaretlenen
veri tabanları için \textbf{Default Schema} sekmesi altına oluşturulan kullanıcı ismi yazılır. Daha sonra OK
seçeneği tıklanır ve devam edilir.

\vspace{0.2cm}
Server’da Databases bölümü genişletilerek kullanıcıya erişimi açılacak olan database sağ tıklanıp
properties seçilir. Ardından oluşturulan kullanıcı seçilir ve aşağıdaki bölümden gerekli izinler verilir.
Bütün bu işlemler bittikten sonra OK seçeneği tıklanır. \textbf{SQL Server yeniden başlatılır.}

\vspace{0.2cm}

Artık SQL Management Studio aracılığı ile veya ftp-telnet ile bağlantı kurulabilir.
$$\textbf{Örnek; telnet 10.0.0.17 1433}$$

\vspace{0.5cm}

\textbf{2.} VLAN nedir? Hangi koşullarda VLAN kullanmak gereklidir. Örnek bir VLAN kurulumu için
kullanılması gereken tüm cihazları ve yapılması gereken ayarları anlatınız. 

\vspace{0.3cm}

\textbf{Sanal yerel alan ağı} anlamına gelen VLAN, bir yerel ağ üzerindeki ağ kullanıcılarının ve kaynakların
\textbf{mantıksal olarak gruplandırılması} ve switch üzerinde portlara atanmasıyla yapılır.

\vspace{0.4cm}

Birçok faydası bulunan VLAN ile büyük ağ yapılarının daha rahat kontrolünün sağlanması hedeflenir. Bazı
durumlarda güvenlik sebebiyle de böyle bir yapılandırma yoluna gidilebilir. VLAN kullanmak için bazı
sebepler;

\begin{itemize}
  \item Local network' te broadcast mesajlarının yol açtığı karmaşayı minimize ederek trafiği azaltır.
  \item Yerel ağ üzerinde her birime en az bir VLAN ataması yaparak daha yönetilebilir bir network elde
edilebilir.
  \item VLAN blokları arasındaki haberleşmeleri (ip-routing) belirleyerek ağı güvenli hale getirebiliriz.
  \item Fiberoptik yada UTP uplink hatları üzerinde Trunk yöntemi ile birçok VLAN-network' ün iletimini
sağlamak amacıyla kullanılabilir/
\end{itemize}

\pagebreak

\begin{figure}[htb]
\begin{center}
\includegraphics[]{a2.png}
\end{center}
\caption{Cisco Switch İle VLAN kurulumu}
\label{fig:Sistem}
\end{figure}

\begin{multicols}{2}
  $$\textbf{Switch 1 Ayarları}$$
  \begin{itemize}
    \item SW1 > enable
\item SW1\# configure terminal
\item SW1(config)\# vtp mode server
\item SW1(config)\#vlan 2 (Bu komutu yazdığınızda VLAN’ınız yaratılmış olacaktır)
\item SW1(config-vlan)\#name yeni (VLAN’ımıza atayacağımız isim)
\item SW1(config-vlan)\#interface fa0/11
\item SW1(config-if)\#switchport access vlan 2
\item SW1(config-vlan)\#interface fa0/12
\item SW1(config-if)\#switchport access vlan 2
\item SW1(config-if)\# (ctrl+z) tuşlarını kullanarak enable mode dönelim.
  \end{itemize}

\columnbreak

  $$\textbf{Switch 2 Ayarları}$$
  \begin{itemize}
    \item SW2\# configure terminal
\item SW2(config)\#vtp mode transparent
\item SW2(config)\#interface fa0/13
\item SW2(config-if)\#switchport access vlan 2 
  \end{itemize}

\end{multicols}

\vspace{1cm}

  \textbf{İki Switch Arası Trunk Ayarları}
  \begin{itemize}
    \item SW1\# configure terminal
\item SW1(config)\# interface fa0/24
\item SW1(config-if)\#switchport mode trunk
\item SW1(config-if)\# (ctrl+z) tuşlarını kullanarak enable mode dönelim.
  \end{itemize}


\pagebreak

\textbf{3. } Satış ve yönetim gibi iki alt birimde mevcut olarak 100 çalışanı çalışanı bulunan bir firma için
aşağıdaki tabloyu kullanarak yerel alan ağını, alt ağlara bölünüz. Alt ağlara ait ağ adresi,
broadcast adresi, en düşük ve en yüksek IP adreslerini hesaplayınız. (IP adresi olarak
192.168.3.x kullanınız)



\begin{figure}[htb]
\begin{center}
\includegraphics[]{a3.png}
\end{center}
\caption{İlgili tablo}
\label{fig:Sistem}
\end{figure}

\vspace{0.2cm}

$$\text{İki alt birim olduğu için iki alt ağa bölebiliriz.}$$
$$\text{Alt Ağ Maskesi 255.255.255.128 }$$
$$\text{11111111.11111111.11111111.10000000 /25 networks 2}$$

\vspace{1cm}

\begin{table}[!htpb]
\centering
\begin{tabular}{@{}lccc@{}}
\toprule
\multicolumn{4}{c}{\textbf{Birinci Ağ}} \\ \midrule
\multicolumn{1}{c}{\textbf{\begin{tabular}[c]{@{}c@{}} Alt Ağ Adresi \end{tabular}}} & \textbf{\begin{tabular}[c]{@{}c@{}}En Düşük IP Adresi\end{tabular}} & \textbf{\begin{tabular}[c]{@{}c@{}}En Yüksek IP Adresi\end{tabular}} & \textbf{Broadcast IP Adresi} \\ \midrule
192.168.3.0 & 192.168.3.1 & 192.168.3.126 & 192.168.3.127 \\
\end{tabular}
\end{table}


\begin{table}[!htpb]
\centering
\begin{tabular}{@{}lccc@{}}
\toprule
\multicolumn{4}{c}{\textbf{İkinci Ağ}} \\ \midrule
\multicolumn{1}{c}{\textbf{\begin{tabular}[c]{@{}c@{}} Alt Ağ Adresi \end{tabular}}} & \textbf{\begin{tabular}[c]{@{}c@{}}En Düşük IP Adresi\end{tabular}} & \textbf{\begin{tabular}[c]{@{}c@{}}En Yüksek IP Adresi\end{tabular}} & \textbf{Broadcast IP Adresi} \\ \midrule
192.168.3.128 & 192.168.3.129 & 192.168.3.254 & 192.168.3.255 \\
\end{tabular}
\end{table}